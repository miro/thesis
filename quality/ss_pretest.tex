 \subsection{Pretest methods}

Capers Jones suggests that in every software project there should be multiple pretest QA methods used. Jones lists a combination of methods for both small and large projects. A small software project is described to have a maximum amount of 1000 function points or 50 000 source code statements. 

These small projects are generally executed by a team with less than 6 software developers. These teams have no specialists for any quality methods, but the developers are generalists handling requirements, design, coding and testing. In many cases with Agile approach, there are users representative embedded in the team providing requirements and customers viewpoint in real time. Jones reminds that removing defects with high efficiency requires trained and technically skilled software engineers instead of generalists. However, this is not so necessary in small projects, since fortunately these projects have usually low defect potentials.

The origins of defects in small studied projects are split into five categories. Source code is the most common origin of defects. About 1.75 defects per function point are found from source code and this leads to 1750 defects in whole projects. Software design is the second most common source of defects. Design is the origin of 1 defects per function point. Requirements are causing 0.75 defects per function point and documentation nearly as much with 0.65 defects. Poorly executed fixes are the origin of 0.27 defects per function point. All together these five sources are the source for 4420 defects in a whole software project. These figures represent the approximate averages and the actual values can be as much as 25\% lower or higher for every source. [TODO: C Jones: Small projects p. 196]





-Pretest
Desk checking p. 208
Peer review p. 209
Scrum sessions p. 222
Static analysis p. 267

-ROI: Three main activities: Review, process audit and testing
