
 \section{Team and Phases}

% http://www.qsm.com/resources/function-point-languages-table
% Function points

% ensimmäinen MVP
% pieni ja yksinkertainen
% kokeilut, testausta siirrettiin tulevaisuuteen
% featuret edellä, testaus jälkikäteen
% asioiden validointi, epäselvät speksit -> tuntui turhalta testata ilman tietoa onko toiminnallisuus järkevää



% -Kerrotaan prosessista ja tiimistä 



% * Vaihe 1
% 	* Aikataulu: 20.11.2013-15.2.2013
% 	* Budjetti: 66 690€ (60 000€)				26%
%	* Function points 				130 (+130)
% 	* Pääkontentti: 
% 		* Lasten ja hoitajien läsnäolon mobiilikirjaus
% 		* Admin UI päiväkotien, ryhmien, lapsien ja hoitajien tietojen syöttämiseen
% 	* Tiimi
% 		* Osmo
% 		* Mike
% 		* Antti
% 		* Jouni
% 		* Elice
% * Vaihe 2-3
% 	* Aikataulu: 19.2.2013-4.6.2013
% 	* Budjetti: 56%   144 840€ (136 000€)
%	* Function points 				415 (+285)
% 	* Pääkontentti: 
% 		* Kommunikointi päiväkodin ja kodin välillä
% 		* Hoitosuunnitelmat
% 		* Päiväkodin käyttöliittymä tietojen selaamiseen ja muokkaukseen sekä raportointiin 
% 	* Tiimi
% 		* Osmo
% 		* Mike
% 		* Kalle
% 		* Mikko (huhtikuu ->)
% 		* Elice
% * Vaihe 4
% 	* Aikataulu: 5.6.2013 - 15.7.2013
% 	* Budjetti: 6%   15 000€
%	* Function points 				475 (+50)
% 	* Pääkontentti: 
% 		* Perhepäivähoidon toiminnot ja rajapinnat 
% 	* Tiimi
% 		* Osmo
% 		* Mike (osa-aikainen)
% 		* Miro
% 		* Kimmo
% 		* (Elice)
% * Vaihe 5
% 	* Aikataulu: 15.7.2013 - 16.7.2013
% 	* Budjetti: 12%   30 000€
%	* Function points 				500 (+25)
% 	* Pääkontentti: 
% 		* Läsnäolotietomallin uudistus, läsnäolotietojen esitys ja muokkaus, kulukorvaukset kunta UI:hin 
% 	* Tiimi
% 		* Osmo (osa-aikainen)
% 		* Mike
% 		* Miro
% 		* Kimmo
% 		* (Elice)



 \section{QA Methods and Processes Used}



% Alkuun testattiin pelkästään UI:n kautta
% Testaus hiipui projektin myötä -> ei “riittävää” testausta
% Bäkkärin integraatiotestit -> testattiin kriittisimpiä osia, muttei kaikkia
% Satunnaisia testejä
% Automaattiset testit jumiin CI:ssä -> aikaa käytettiin korjaukseen, mutta ei onnistunut -> Testit % pois käytöstä
% Testit koetaan tärkeiksi
% laajuus kasvaa, joten ei tiedä miten asiat toimii 
% regressio uusien featureiden kanssa
% turvaverkko
% tekijät vaihtuu
% testit valmiita esimerkkejä oikeasta käytöstä/toiminnasta
% testien dokumentointi, ei tiedetty mitä testit testaa -> väärä luottamus
% TDD käytetty satunnaisesti bugien korjauksissa

 
% -Käytetyt QA-metodit
% -Static code analysis (IDEn vakiot)
% -Communication ("embedded users")
% -Testing(frontend, backend unit testing and integration tests)
% -Occasional reviews (rarely)
% -Continuous Integration


 \section{Achieved Quality in The Project}

% bugeja huomattiin, jotka olisi löytynyt testeillä
% Tiimin mielestä testien kirjoituksella olisi säästänyt aikaa/rahaa
% Testiympäristö ei ollut vastaava tuotannon kanssa, muistinvarainen vs HDD, Eri tietokanta
% Bugeja tietokannan käsittelyssä, jotka ilmaantuivat vain tietyssä kannassa
% Skaalaus
% Päiväkotien määrän kasvaessa suorituskyky laski eksponentiaalsesti
% Aiemmin ei tunnettu tarvetta optimoinnille
% indeksi
% tiedonhaku yms.
% Ei mikään varsinainen ongelma, mutta yllätti


% Featuree featuren perään, asiakkaan puolelta
% weppijutuissa nopeasti näkyvää asiaa “huonosti”
% Tahti liian kova
% tekninen velka -> realisoituu myöhemmin
% Refaktoroinneille ei aikaa
% Kunnon tekeminen vaatii aikaa, ei ratkea “rahalla”
% Featureiden toteutus hidastuu eksponentiaalisesti kun tehdään pitkään nopeasti (ja huonosti)
% Asiakas ei ymmärrä mitä aiemmat “oikopolut” tarkoitta
% Asiakas ei muista teknistä velkaa
% Kommunikaatio aluksi tae hyvälle laadulle
% Aluksi ongelmallisin kommunikaatio asiakkaan kanssa
% Myöhemmin kommunikaatio failannut tiimin sisälläkin, koska osa-aikaisia samassa projektissa
% 
% Mitä erilailla
% Testit toimimaan
% UI
% Järjestelmällisemmät testit
% Definition of Done selkeämmäksi: testit kuuluu taskiin
% Rajapinnasta tarkempi määrittely
% “REST”, mutta muita tapoja seassa
% Mietitty puhdas REST vs clientille tehty API
% Clientin helppo noutaa datoja
% Requestien määrän minimointi
% Dokumentointi kuntoon
% Rakenteellinen laatu kuntoon
% Aikavyöhykeasiat kerralla kuntoon / päätös tehdä ilman monimutkaisuutta
% Samaan aikaan samalla tiimillä nykyisen järjestelmän ylläpitoa ja uudet featuret
% Uudet featuret “suoraan” nykyiseen järjestelmään
% 
% Tulevaisuudessa paremmin
% järjestelmällisyys
% yhdessä sovitut prosessit
% Pelisäännöt, ei byrokratiaa
% Auttaisi jos ja kun porukka vaihtuu
% kaikki koodarit samojen käytäntöjen alle, myös koodaavat asiakkaat
% laatuasioista “valittaminen” ei henkilökohtaista
% pitäisi sopia etukäteen, että saa sanoa mistä vaan asiasta loukkaantumatta
% review, pullrequestit
% todettu hyviksi nyt kun käytössä
% pitäisi osata sanoa koodaavalle asiakkaalle, että osa siitä aiheuttaa lisätyötä tiimille
% Kommunikaatioo paremmaksi








% -Kartoitetaan bugit (Mailit, Pivotal, Repo)  
% -Asiakkaan ja loppukäyttäjien tyytyväisyys
% 
% Miken lista hyvästä laadusta:
% -I Know It When I See It -> Validoinnit
% -Lyhyet sprintit
% -Tasainen vastuunjako
% -Kommunikointi
% -Refaktoroinnit
% -MVP





% Juttelut
% 
% -Isot teemat:
% -testaus
% -kiire/featurekeskeisyys
% -MVP -> tuotteeksi ilman tuotteistusta
% -Rakenne, API