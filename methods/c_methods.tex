
\chapter{Methods}
Tämä kappale on itselleni kaikista hämärin, että mitä täällä pitäisi olla

\section{Design Research}
Erityisesti tämä osio

\section{Case: Päikky-system and Offline-mode}
Tänne jotain tavaraa Päikky-toimitusjohtajan haastattelusta (jota ei siis vielä ole tehty?). Ehkä jos jotain löytyy kommunikaatiosta/kehitysdokumenteista niin se myös?

Kannattaisiko tänne kaivaa versionhallinnasta dataa offlinemoodista, tyyliin millä aikavälillä se kehitettiin, paljonko arvioitu rivien määrä, etc?

\subsection{Before and after Offline-mode}
Softan käytöstä ennen offline-modea

Softan käytöstä offline-moden jälkeen


% Implementaatiosta sen verran että tutkimusmetodina on implementoida teknologiaa ja tutkia kuinka se soveltuu arkipäivän käyttöön. Voisi myös miettiä lähestymistä että sulla on ollut kaksi eri tilannetta 1. softa ilman offline"palikkaa/featurea" ja 2. offline palikan kanssa. Tutkit että kuinka offlinepalikka vaikutti sovelluksen käyttökokemukseen.

\subsection{Validating Offline-mode User Experience with User Interviews}
Avattu, keneltä on kysytty (päiväkodin tädit) mitä (Päikky-järjestelmän käyttöä ja offline-moodin käsittämisestä juttua) ja miksi (pyritään ymmärtämään onko offline-mode vastaus todelliseen käyttäjien ongelmaan). Datan keräysmetodina puolistrukturoitu käyttäjähaastattelu.
% datan keräysmetodina käyttäjä/asiakashaastattelut (muuta?, kehitysdokumentit?)

  


