\thispagestyle{empty}
 
\vspace*{-.5cm}\noindent
 
% If thesis is in English, use the file "tut-logo"
% instead of "tty-logo" in the following:
\includegraphics[width=8cm]{assets/tut-logo}
 
\vspace{6.8cm}
 
\noindent{\bf\large \textsf{Miro Nieminen}}\\
{\bf\large \textsf{Fallback Mechanisms for Connection Loss in Single-Page Web Applications}}\\
\textsf{Master of Science Thesis}
 
\vspace{6.7cm} % jos kaksi otsikkoriviä vaihda -> 6.7cm, muutoin 8.7cm
 
\begin{flushright}
  
\begin{minipage}[c]{6.8cm}
\begin{spacing}{1.0}
\textsf{Examiners: Tommi Mikkonen}\\
\textsf{Examiners and topic approved in}\\ 
\textsf{the Information Technology}\\
\textsf{Department Council meeting on}\\
\textsf{xx.02.2015}\\
\end{spacing}
\end{minipage}
\end{flushright}



%% ----------------------------------------
\newpage
 
\setcounter{page}{1} % tämä tarvitaan, jottei ensimmäinen sivu kansilehden jälkeen olisi numero 2.
 
\chapter*{TIIVISTELMÄ}
\begin{spacing}{1.0}
\textsf{TAMPEREEN TEKNILLINEN YLIOPISTO}\\
\textsf{Tietotekniikan koulutusohjelma}\\
{\bf \textsf{Miro Nieminen: Fallback Mechanisms for Connection Loss in Single-Page Web Applications}}\\
\textsf{Diplomityö, 53 sivua}\\
\textsf{Maaliskuu 2015}\\
\textsf{Pääaine: Ohjelmistotuotanto}\\
\textsf{Tarkastajat: Tommi Mikkonen}\\
\textsf{Avainsanat: JavaScript, Web Application, Single-Page Application, Computer Supported Collaborative Work, Offline Support}\\
\end{spacing}
 
Web-teknologioiden nopea kehittyminen niin työpöytä- kuin mobiililaitteissa on tehnyt selaimesta varteenotettavan sovellusalustan lähes kaikenlaisille ohjelmistoille. Suorituskykyisen selaimen löytyminen yhä useammasta taskusta tekee web-tek\-no\-lo\-gi\-oi\-den käyttämisestä yhä houkuttelevampaa ja kustannustehokkaampaa myös toteutettaessa liiketoimintakriittisiä sovelluksia.

Mobiililaitteiden määrän kasvuvauhti on ohittanut matkapuhelinverkkojen datayhteyksien kantokyvyn kasvuvauhdin. Laitteita myös käytetään yhä syrjäisemmissä sijainneissa, missä datayhteydet ovat rajoittuneita tai jopa olemattomia. Tämä aiheuttaa ongelmia käytettäessä web-sovelluksia, joiden toiminta on riippuvainen yhteydestä palvelimeen. Huono datayhteys tuottaa ongelmia web-sovelluksen käyttäjälle, kun sovellus saattaa olla hetkittäin täysin toimimattomassa tilassa.

Tässä työssä keskitytään siihen, miten web-kehittäjät voivat varautua yhteyden katkoksiin ja heikkoon laatuun sovellustasolla. Työn pohjana käytetään tapaustutkimusta, missä päiväkotiympäristössä käytettävään Päikky-sovellukseen lisätään offline-tuki. Tapaustutkimuksessa toteutettuja ratkaisuja arvioidaan käyttäjähaastattelujen avulla sen kannalta, miten ne sopivat yleispäteviksi ratkaisumalleiksi web-sovelluksissa yhteyskatkosten varalle. Tehtyjä ratkaisuja arvioidaan myös käyttökokemusnäkökulmasta, ja siitä, miten ne tukevat ja ovat ymmärrettävissä keskivertokäyttäjälle.

Työn tulosten pohjalta esitetään suunnitteluperiaatteita sovelluskehittäjille yh\-te\-ys\-kat\-kok\-siin varautumista varten. Käyttäjähaastatteluissa ilmentyneet ratkaisun käyttökokemuksen epäkohdat listataan, ja niihin ehdotetaan mahdollisia parannusehdotuksia ja jatkotoimenpiteitä.

Työn keskeisimmät tulokset osoittavat, että yhteysongelmiin tulee ainakin jollakin tasolla varautua nykypäivänä kaikissa vakavastiotettavissa web-sovelluksissa. Vaikka varsinaista offline-tukea ei toteutettaisikaan, voidaan työssä esitetyillä suunnitteluperiaatteilla parantaa huomattavasti web-sovelluksen käyttökokemusta.






%% ----------------------------------------
\newpage
\chapter*{ABSTRACT}
\begin{spacing}{1.0}
\textsf{TAMPERE UNIVERSITY OF TECHNOLOGY}\\
\textsf{Master's Degree Programme in Information Technology}\\
{\bf \textsf{Miro Nieminen: Fallback Mechanisms for Connection Loss in Single-Page Web Applications}}\\
\textsf{Master of Science Thesis, 53 pages}\\
\textsf{March 2015}\\
\textsf{Major: Software Engineering}\\
\textsf{Examiner: Tommi Mikkonen}\\
\textsf{Keywords: JavaScript, Web Application, Single-Page Application, Computer Supported Collaborative Work, Offline Support}\\
\end{spacing}
 
 % Konteksti
\noindent
Here will be dragons
 
 % Ongelma
\noindent

 % Tässä diplomityössä (mitä tehty ratkaisemiseksi)
\noindent

% Arvio onnistumisesta
\noindent






%% ----------------------------------------
\newpage
 
\chapter*{PREFACE}
\noindent 

Academic writing surely is not for everyone.

Your advertisement could have been here.




%% ----------------------------------------
\newpage
\tableofcontents
%\newpage
%\listoffigures
%\listoftables










%% ----------------------------------------
\newpage
\renewcommand{\chaptermark}[1]{\markboth{\thechapter. \ #1}{}}
\renewcommand{\sectionmark}[1]{\markright{}{}}
\lhead{\fancyplain{}{\leftmark}}
 
