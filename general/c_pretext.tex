e 
\thispagestyle{empty}
 
\vspace*{-.5cm}\noindent
 
% If thesis is in English, use the file "tut-logo"
% instead of "tty-logo" in the following:
 
\includegraphics[width=8cm]{assets/tut-logo}
 
\vspace{6.8cm}
 
\noindent{\bf\large \textsf{Miro Nieminen}}\\
{\bf\large \textsf{}}\\
\textsf{Master of Science Thesis}
 
\vspace{8.7cm} % jos kaksi otsikkoriviä vaihda -> 6.7cm
 
\begin{flushright}
  
\begin{minipage}[c]{6.8cm}
\begin{spacing}{1.0}
\textsf{Examiners: Tarkastaja 1}\\
\textsf{Examiners and topic approved in}\\ 
\textsf{the Information Technology}\\
\textsf{Department Council meeting on}\\
\textsf{xx.xx.xxxx}\\
\end{spacing}
\end{minipage}
\end{flushright}
 
\newpage
 
\setcounter{page}{1} % tämä tarvitaan, jottei ensimmäinen sivu kansilehden jälkeen olisi numero 2.
 
\chapter*{TIIVISTELMÄ}
\begin{spacing}{1.0}
\textsf{TAMPEREEN TEKNILLINEN YLIOPISTO}\\
\textsf{Tietotekniikan koulutusohjelma}\\
{\bf \textsf{Miro Nieminen: <otsikko>}}\\
\textsf{Diplomityö, xx sivua}\\
\textsf{Xxxxxkuu 2015}\\
\textsf{Pääaine: }\\
\textsf{Tarkastajat: }\\
\textsf{Avainsanat: }\\
\end{spacing}
 
\noindent
Tekstiä
 
\noindent
Tekstiä

\newpage
\chapter*{ABSTRACT}
\begin{spacing}{1.0}
\textsf{TAMPERE UNIVERSITY OF TECHNOLOGY}\\
\textsf{Master's Degree Programme in Information Technology}\\
{\bf \textsf{Miro Nieminen: <otsikko>}}\\
\textsf{Master of Science Thesis, 61 pages}\\
\textsf{xxxxxx 2014}\\
\textsf{Major: }\\
\textsf{Examiner: }\\
\textsf{Keywords: }\\
\end{spacing}
 
 % Konteksti
\noindent
Tekstiä
 
 % Ongelma
\noindent
Tektsiä

 % Tässä diplomityössä (mitä tehty ratkaisemiseksi)
\noindent
Tekstiä

% Arvio onnistumisesta
\noindent
Tektsiä

\newpage
 
\chapter*{PREFACE}
\noindent 

Tässä on tapana vaan kiitellä kaikkia. Tämä nimenomainen templaattiin laitettu teksti on Mikko Pohjan mukaan "ihan tyhmä".
 
\newpage
\tableofcontents
%\newpage
%\listoffigures
%\listoftables

% \newpage
% \chapter*{TERMS AND DEFINITIONS}
%  
% % Näitä ei välttämättä tarvi olla
% \begin{termiluettelo}
%  
% \item [Life cycle] TODO
% \item [Function point] TODO
%  
% \end{termiluettelo} 
 
 
\newpage
\renewcommand{\chaptermark}[1]{\markboth{\thechapter. \ #1}{}}
\renewcommand{\sectionmark}[1]{\markright{}{}}
\lhead{\fancyplain{}{\leftmark}}
 
