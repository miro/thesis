
\chapter{Implementation}
This chapter coverages what was done to the Päikky system in order to get it working also during the Internet connection shortages, and how that was done. 

%% --------------

\section{Basic Principles of the Päikky Offline-mode}
blabla

\subsection{The goal of the Offline-mode}
Mitä Offline-modella pyrittiin saavuttamaan

\subsection{Limited Feature Set}
Offline-moden rajoitettu featuresetti - mitä ja miksi

\subsection{Actions when Entering and Leaving Offline mode}
Yleisellä tasolla mitä sovellus tekee kun mennään offline-modeen, ja mitä kun tullaan sieltä pois

%% --------------
\section{Technical Details}

\subsection{HTML Cache Headers}
Apacheconffauksista, otsikko saattaa vaatia työtä


\subsection{Application Cache}
Cache manifestista juttua, perusteet, ja se, miten Päikky sitä käyttää

\subsection{Identifying Bad Connectivity}
Millä mekanismilla yhteyden häviäminen tunnistetaan


\subsection{Local Storage}
Local stroagen perusteet

REST-API-pyyntöjen cachettamisesta local storageen juttua


\subsection{Promise-based Presence Marking queue}
Kuinka koodin tasolla on toteutettu tehtävien merkkauksien jonosysteemi ja sen purkaminen

(Perus)Teoriaa Promiseista JavaScriptissä


%% --------------
\section{Known Limitations of the Offline mode}
Esimerkiksi konfliktitilanteita nykyisessä ratkaisussa ei huomioida/niihin ei varauduta mitenkään (esimerkiksi jos kaksi käyttäjää yhtäaikaa editoi offlinessa juttuja ja laitteet menevät onlineen -> mitä tapahtuu on mysteeri. Tätä on paikattu käyttäjäkoulutuksella, että eivät aja järjestelmää tällaisiin tilanteisiin)


(Onkohan tämä sittenkin turha kappale, tai väärässä paikassa? Olisiko yhdistettävissä evaluation-kappaleeseen?)

