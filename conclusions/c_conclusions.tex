
 \chapter{Conclusions}

 % kerrottiin softa-startupeista yleisellä tasolla
 % listattiin perinteisen softakehityksen laatumetodit eri elinkaaren vaiheissa sekä metodien tehokkuudesta
 % kerrottiin laadusta yleisesti sekä startupin näkökulmasta
 % kerrottiin startupin laatukäsityksestä ja siitä miten hyvää laatua voi tavoitella
 % yhdistettiin perinteisen softakehityksen laatumetodeita startupin ympäristöön
 % käsiteltiin case-projektin suoritus ja vaiheet 
 % käsiteltiin case-projektin toteutunut laatu haastattelujen perusteella
 
 In this thesis, quality improvement in a software startup environment was studied. Startup environment was examined from the viewpoint of Lean Startup methodology. Software quality was described both in traditional sense and in a software startup context. Thesis defined and evaluated software improvement methods and activities that were then linked into the startup environment and its view of software quality.

 A case project was also presented by describing its execution and phases in detail. Also the quality improvement methods used in the project were described. Quality achieved in the project was assessed with the data from interviews and some suggestions for improving the quality and the development processes were mentioned.


 % ---Mitä opittiin/tehtiin---
 % sekä perinteisessä että modernissa softakehityksessä painotus laadun parannuksessa pitäisi olla virheiden estämisessä enemmän kuin niiden löytämisessä ja korjaamisessa
 % moderni startupin softakehitys koostuu lyhyistä iteraatioista ja muista agile-jutuista, joten perinteinen laatudvarmistus ei ole suoraan sovellettavissa startupiin
 % perinteiset laatumetodit on kuitenkin sellaisia, joita voi käyttää startupin modernissa kehityksessä
 % Perinteinen laadunvarmistus keskittyy enemmän teknisiin asioihin, kun modernissa kehityksessä fokus on enemmän ihmisissä ja tekemisen mallissa

Improving quality in a software startup development emphasizes the importance of defect prevention over defect removal. This is common with traditional software development. In traditional software development though, the effort put to defect removal activities and testing has larger share of total effort than in startup environment. Another difference between these two environments is that in modern startup environment, the development is usually done in short iterations. The patterns in traditional development are not applicable in this kind of development as such, but the activities can be applied by adjusting them for use in short iterations.

When traditional quality improvement has more focus on the technical aspects, methodologies of modern software development quality focus on the people and processes of development. Both the team and individuals are considered important and the emphasis is on the motivation, expertise and leadership of the personnel. The development should include plenty of communication and the team should have all the resources and authority to make all the important decisions. The technical aspects to consider in software startups are the integrity of the product and the iterative development containing constant improvement.
