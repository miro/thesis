
\chapter{Related Research}

<Yleiskuvausta tutkimuksesta>

(Mieleni tekisi nimetä tämä kappale "Backgroundiksi", olisikohan parempi?)

% Yleisesti verkoista, tekniikasta, toimivuusta, ratkaisuja offline moodiin

\section{Single-Page Applications}
SPA's in general

REST API:sta jotain (Päikky cachettaa nimenomaan tämäntyyppisiä requesteja)

"Frontend gets fatter and fatter" -> Päikyn tilakone myös frontendissä nyt


\section{Problems in Internet Connectivity}
 - eri tavat yhdistää internetiin (piuhayhteys, wlan, mobiilidata)
 - kännyverkon toiminnasta jotain?
 - kännyverkon toimivuuden vaihtelun yleisyydestä? 

 \section{Computer Supported Collaborative Work}
 "Reaaliaikaisesta informaationjaosta päiväkotiympäristössä"
% - verkkovälitteinen vuorovaikutus
% - hakusanalla computer supported collaborative work (CSCW) löytyy kamaa 

\section{Existing Solutions for the Offline Problem}
Minkälaisia offline ratkaisuja on olemassa web/känny-puolella (miten google/microsoft yms. ovat lähestyneet ongelmaa)?


\section{Research Gap}
% -> Gappiin ehkä jotain, että verkkovälitteistä vuorovaikutusta on tutkittu paljon (katso CSCW kamaa) ja että mobiiliverkojen tutkimus/kehitys/käyttö ollut räjähdymäistä. Kuitenkaan ei ole paljoa tutkimusta siitä kuinka verkon katkeaminen vaikuttaa reaalisaikaista tiedonvälitystä vaativan sovelluksen käyttökokemukseen ja kuinka sovellussuunnittelussa tulisi katkot ottaa huomioon.  
