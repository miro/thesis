
\chapter{Related Research}

<Yleiskuvausta tutkimuksesta>

% Yleisesti verkoista, tekniikasta, toimivuusta, ratkaisuja offline moodiin


\section{Basics of Mobile Device Data communication}
% - kännyverkon toiminnasta jotain?
% - kännyverkon toimivuuden vaihtelun yleisyydestä? 

\section{Existing Solutions for the Offline Problem}
% - minkälaisia offline ratkaisuja on olemassa web/känny-puolella (miten google/microsoft yms. ovat lähestyneet ongelmaa

\section{Computer Supported Collaborative Work}
% Erityisesti liittyen sun caseen (reaaliaikainen informaationjako, päiväkoympäristö?)
% - verkkovälitteinen vuorovaikutus
% - hakusanalla computer supported collaborative work (CSCW) löytyy kamaa 


\section{Research Gap}
% -> Gappiin ehkä jotain, että verkkovälitteistä vuorovaikutusta on tutkittu paljon (katso CSCW kamaa) ja että mobiiliverkojen tutkimus/kehitys/käyttö ollut räjähdymäistä. Kuitenkaan ei ole paljoa tutkimusta siitä kuinka verkon katkeaminen vaikuttaa reaalisaikaista tiedonvälitystä vaativan sovelluksen käyttökokemukseen ja kuinka sovellussuunnittelussa tulisi katkot ottaa huomioon.  
