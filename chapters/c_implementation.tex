
\chapter{Implementation}
This chapter coverages what was done to the Päikky system in order to get it working also during the Internet connection shortages, and how that was done. 

This chapter is concerned with the facts of what the offline mode consists of and how it was implemented. Reasoning for the decisions made are also explained. Analyzing the consequences of those decisions are done later on the evaluation chapter.

%% --------------
\section{The goal of the Offline mode}
% Mitä Offline-modella pyrittiin saavuttamaan?
The primary goal for the offline mode on Päikky was to enable the most critical tasks for kindergarten nurses on all situations [?petehaastattelu?] on the Kindergarten UI (the mobile friendly version of the Päikky system). Päikky's main and the most important feature is the ability to track the attendance of the kindergarten's children on real time. In order to achieve that, nurses must be able to mark the children's coming and going without getting interrupted by the limitations of the application. The offline mode was implemented in order to remove the network status related limitations on the key activity mentioned above. Nurses should be able to use the children logging feature on Päikky under any kind network condition. If the network connection is nonexistent, the application should record user's actions and save them to the server once the network connection is achieved again.

Secondary goal for the offline mode were to allow as seamless usage as possible of the Päikky on kindergartens where network connection is weak. Nurses should be able to see the information from Päikky even if there is no network connection at the time. Information should be served based on the best-effort delivery: application should show all the information it has at the time to the user while trying to fetch the most latest version of the information.

The ultimate vision - yet to be reached - for the offline mode is that it could abstract the network quality completely away from the user's consideration. Under any network quality the user should be able to use Päikky normally without any interference. On the current version this is not yet achieved nor scoped to be achieved. 




%% --------------
\section{State transitions to and from the Offline mode}
% offline ja online -moodin selitykset, termien avaus? erityisesti "online-moodi" (jos online-moodia ei selitetä, niin sitten ainakin pitää muissa kappaleissa puhua jostain muusta kuin online-moodista)

% HOX onko päällekkäisyyttä methods-case -kuvauksen kanssa?

 





%% --------------
\section{Limited Feature Set}
% Offline-moden rajoitettu featuresetti - mitä ja miksi
Similar to every other real life software project also in Päikky compromises has been made while balancing between the scope, resources and quality. In order to create software with good quality under given budget, the scope had to be kept reasonable and some prioritization between features had to be made. This resulted the first version (the one studied by this thesis) to be technically quite simple and even naive on some aspects. Users experience this as a lack or disabling of some features on the offline mode.




%% ###
\subsection{Disabled features}
When the application enters offline mode, every feature but the critical key functionality are disabled for the user. These include features such as

\begin{enumerate}
    \item sending messages on the application,
    \item editing persons' information,
    \item changing persons' photos,
    \item editing existing presence markings or upcoming presence plans.
\end{enumerate}

These features were left out from the offline support based on feature importance evaluation done by the customer [?todo-hox petehaastis?]. Since under available budget some features had to excluded from the offline mode, the ones listed above were selected due to the nature of their basic use case.

None of the listed activities are essential for the Päikky's key feature, the real time tracking of kindergarten's personnel attendance. If there is no Internet connection available at the time, each one of these activities can be postponed without sacrificing the integrity of the presence data on the system until the Internet connection is available again.




%% ###
\subsection{Simplified Data Synchronizing}
% synkkaus-yksinkertaistus-juttui
% nonexistent collision check!
% googlewavepaprusta jotain dadaaa tänne?
The nature of the Päikky usage by the nurses allowed the development team to make some simplifications to the implementation of the offline mode. These simplifications included the way how conflicts between concurrent changes are solved. To put it bluntly, they aren't solved in any way.

To understand why this solution was feasible, one has to understand the environment and practices about how Päikky is operated. The usual scenario on the kindergartens where Päikky is used is that each kindergarten group has only one, dedicated device. With minor exceptions all of the presence markings for the care group's personnel are done via the group's dedicated device, which is also the only device for the group to access Päikky. This is also the way how usage of Päikky is instructed by the MukavaIT[todo-hox miten mukavait:n viitataan?] to the kindergarten personnel. Editing person's presence status from different devices while on offline mode is especially discouraged. With these guidelines the probability for cases where two different devices have made concurrent changes to individual person becomes almost non-existent. [?todo-hox tähän Pete-haastiksesta vahvistus?]

These circumstances allowed the development team to streamline the data synchronization on the Päikky server. On agreement with MukavaIT, there is no functionality that tries to solve possible merge conflicts on the presence data. If there appears a situation where two devices have concurrently changed the attendance status of a single person - contrary on the instructions given to the users - both of the changes are saved to the database. Fixing the data to reflect the situation happened in the real world is left to the responsibility of the kindergarten personnel. 
% backend-offline-tuen olemattomuus myös tässä?

This approach removes the need of offline related functionality on the Päikky server almost completely. By this the development effort was cut significantly: the probability of creating data conflicts has been made minimal with the user instructions, and in the implausible case of a conflict to appear resolving it is left to the responsibility of the user, not by the code base. 

From the server point of view, presence markings done on the offline mode are received and stored exactly same way as are the markings done real time on the online mode. Only thing that differs is the time gap between the presence marking's timestamp and the occasion when the presence marking is received on the server. This is described in detail on the section <Storing and Receiving Presence M...>[todo-hox]. % tämä ehkä muualle?





%% --------------
\section{Technical Details}
% HOX tää koko setti on lähinnä frontendissä, backendiin ei juurikaan tehty mitään lisätemppuja offline-moodia varten??

%% ###
\subsection{HTML Cache Headers}
Apacheconffauksista, otsikko saattaa vaatia työtä

%% ###
\subsection{Application Cache}
Cache manifestista juttua, perusteet, ja se, miten Päikky sitä käyttää


%% ###
\subsection{Identifying Bad Connectivity}
Millä mekanismilla yhteyden häviäminen tunnistetaan


%% ###
\subsection{Local Storage}
Local storagen perusteet

REST-API-pyyntöjen cachettamisesta local storageen juttua

%% ###
\subsection{Promise-based Presence Marking queue}
Kuinka koodin tasolla on toteutettu tehtävien merkkauksien jonosysteemi ja sen purkaminen

(Perus)Teoriaa Promiseista JavaScriptissä


%% ###
\subsection{Storing and Receiving Presence Markings on the Server}
% ainoa offlineen liittyvä muutos mikä toteutettiin bäkkäriin
% presenceille timestamp, milloin ne on tehty -> voi (ja on) eri kuin mitä vastaanottotimestamp on
% -> yliyksinkertaistaen käytännössä kaikista merkinnöistä tuli "offlinessä-tehtyjä" verrattuna aikaisempaan, missä merkintöihin otettiin timestampit serverille saapumisajan mukaan




%% --------------
% \section{Known Limitations of the Offline mode}
% Esimerkiksi konfliktitilanteita nykyisessä ratkaisussa ei huomioida/niihin ei varauduta mitenkään (esimerkiksi jos kaksi käyttäjää yhtäaikaa editoi offlinessa juttuja ja laitteet menevät onlineen -> mitä tapahtuu on mysteeri. Tätä on paikattu käyttäjäkoulutuksella, että eivät aja järjestelmää tällaisiin tilanteisiin)


% (Onkohan tämä sittenkin turha kappale, tai väärässä paikassa? Olisiko yhdistettävissä evaluation-kappaleeseen?)
% --> toistaiseksi otettu pois, tätä kamaa tuli jo tämän chapterin alussa & lisää tulee (luultavasti) evaluationissa

