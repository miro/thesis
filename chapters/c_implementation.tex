
\chapter{Implementation}
This chapter coverages what was done to the Päikky system in order to get it working also during the Internet connection shortages, and how that was done. 

This chapter is concerned with the facts of what the offline mode consists of and how it was implemented. Reasoning for the decisions made are also explained. Analyzing the consequences of those decisions are done later on the evaluation chapter.

%% --------------
\section{The goal of the Offline support}
% Mitä Offline-modella TAI SIIS SUPPORTILLA pyrittiin saavuttamaan?
% (Pitäisikö yleisemminkin erottaa offline TUKI ja kindergarten-UI:n offline MOODI?)
The primary goal for the offline support on Päikky was to enable the most critical tasks for kindergarten nurses on all situations [?petehaastattelu?] on the Kindergarten UI (the mobile friendly version of the Päikky system). 

Päikky's main and the most important feature is the ability to track the attendance of the kindergarten's children on real time. In order to achieve that, nurses must be able to mark the children's coming and going without getting interrupted by the limitations of the application. In order to offer offline support on the Kindergarten UI, a new feature called \textit{Offline mode} was implemented. Offline mode aims in removing the Internet connection quality related limitations on the nurses' key activity. Nurses should be able to use the children logging feature on Päikky under any kind of Internet condition. If the Internet connection is nonexistent, the application should record user's actions and save them to the server once the Internet connection is achieved again.

Secondary goal for the offline support were to allow as seamless usage as possible of the Päikky on kindergartens where Internet connection is weak. Nurses should be able to see the information from Päikky even if there is no Internet connection at the time. Information should be served based on the best-effort delivery: application should show all the information it has at the time to the user while trying to fetch the most latest version of the information.

The ultimate vision - yet to be reached - for the offline support is that it would abstract the Internet connection quality completely away from the user's consideration. Under any connection quality the user should be able to use Päikky normally without any interference. On the current version this is not yet achieved (nor it was scoped to be achieved). % saisko tähän kappaleeseen jonkin viitteen johonkin käytettävyys-diibadaapaan, johonkin käyttäjän "mielenmalleihin" tms eli siis että ultimate goali olisi, ettei käyttäjän tarvitsisi huolehtia tuosta verkon laadusta ollenkaan, jopa niin että käyttäjä ei edes tietäisi ollaanko online- vai offline-moodissa?




%% --------------
\section{State transitions to and from the Offline mode}
% offline ja online -moodin selitykset, termien avaus? erityisesti "online-moodi" (jos online-moodia ei selitetä, niin sitten ainakin pitää muissa kappaleissa puhua jostain muusta kuin online-moodista)

% HOX onko päällekkäisyyttä methods-case -kuvauksen kanssa?
Implementation of the offline support to the Kindergarten UI made it to have two different states related to the Internet connection status, the already mentioned \textit{Offline mode} when the Internet connection is poor or nonexistent, and the \textit{Online mode} when the Internet connection is working normally. 

The current mode is implicated to the user clearly: while in offline-mode, the Kindergarten UI's header changes color scheme to greyscale and the title says directly "You are working on the offline mode", localized to the user's language.

While in Online mode the Kindergarten UI works almost identically as it did before the offline support implementation. Major changes are that all the API requests done to the Päikky backend are cached to the Local Storage of the device running the Kindergarten UI. Also the sending of Presence marking changes to the backend are done by a dedicated component. Both of these are primarily refactoring the inner parts of the Kindergarten UI, while using the application the user should not experience any difference to the versions prior from these changes.

When entering the Offline mode the method on how data is fetched changes. Instead of fetching the data user requests from the backend, the Kindergarten UI fallbacks to the data cached on the Local Storage. This won't guarantee the availability of the newest information to the user, but at least showing the best effort version is possible. Due to the nature of the Päikky's data, in most of the cases this is acceptable (e.g. when looking for children's parents' phone numbers, and other similar data that is not altered on daily basis). Also some of the features on the Kindergarten UI are disabled when the Offline mode comes active. This functionality is described thoroughly on the <Using Local Storage as a Cache> -section [hox-todo].

While the Offline mode is active, the component responsible for sending the Presence marking changes – the \textit{Job Queue} – acts also differently. If there is a recognized issue with the Internet connectivity (which triggers the Offline mode to be activated), Job Queue stops sending the Presence Markings to the backend but instead saves them to the Local Storage. User is notified on this by showing all the time the size of the queue on the top right corner of the Kindergarten UI. When the Internet connection is active again, the Job Queue starts to send the cached Presence Markings to the backend one at a time. This functionality is explained in depth on the <Job Queue> -section [hox-todo]. 





%% --------------
\section{Limited Feature Set on the Offline mode}
% Offline-moden rajoitettu featuresetti - mitä ja miksi

Similar to every other real life software project also in Päikky compromises have been made while balancing between the scope, the available resources and the quality of the end product. In order to create software with good quality under given budget, the scope had to be kept reasonable and some prioritization between features had to be made. This resulted the first version (the one studied by this thesis) to be technically quite simple and even naive on some aspects. Users experience this as a lack or disabling of some features on the offline mode.




%% ###
\subsection{Disabled features}
When the application enters offline mode, every feature but the critical key functionality are disabled for the user. These include features such as

\begin{enumerate}
    \item sending messages on the application,
    \item editing persons' information,
    \item changing persons' photos,
    \item editing existing presence markings or upcoming presence plans.
\end{enumerate}

These features were left out from the offline support based on feature importance evaluation done by the customer [?todo-hox petehaastis?]. Since under available budget some features had to excluded from the offline mode, the ones listed above were selected due to the nature of their basic use case.

None of the listed activities are essential for the Päikky's key feature, the real time tracking of kindergarten's personnel attendance. If there is no Internet connection available at the time, each one of these activities can be postponed without sacrificing the integrity of the presence data on the system until the Internet connection is available again.




%% ###
\subsection{Simplified Data Synchronizing}
% synkkaus-yksinkertaistus-juttui
% nonexistent collision check!
% googlewavepaprusta jotain dadaaa tänne?
The nature of the Päikky usage by the nurses allowed the development team to make some simplifications to the implementation of the offline mode. These simplifications included the way how conflicts between concurrent changes are solved. To put it bluntly, they aren't solved in any way.

To understand why this solution was feasible, one has to understand the environment and practices about how Päikky is operated. The usual scenario on the kindergartens where Päikky is used is that each kindergarten group has only one, dedicated device. With minor exceptions all of the presence markings for the care group's personnel are done via the group's dedicated device, which is also the only device for the group to access Päikky. This is also the way how usage of Päikky is instructed by the MukavaIT[todo-hox miten mukavait:n viitataan?] to the kindergarten personnel. Editing person's presence status from different devices while on offline mode is especially discouraged. With these guidelines the probability for cases where two different devices have made concurrent changes to individual person becomes almost non-existent. [?todo-hox tähän Pete-haastiksesta vahvistus?]

These circumstances allowed the development team to streamline the data synchronization on the Päikky server. On agreement with MukavaIT, there is no functionality that tries to solve possible merge conflicts on the presence data. If there appears a situation where two devices have concurrently changed the attendance status of a single person - contrary on the instructions given to the users - both of the changes are saved to the database. Fixing the data to reflect the situation happened in the real world is left to the responsibility of the kindergarten personnel. 
% backend-offline-tuen olemattomuus myös tässä?

This approach removes the need of offline related functionality on the Päikky server almost completely. By this the development effort was cut significantly: the probability of creating data conflicts has been made minimal with the user instructions, and in the implausible case of a conflict to appear resolving it is left to the responsibility of the user, not by the code base. 

From the server point of view, presence markings done on the offline mode are received and stored exactly same way as are the markings done real time on the online mode. Only thing that differs is the time gap between the presence marking's timestamp and the occasion when the presence marking is received on the server. This is described in detail on the section <Storing and Receiving Presence M...>[todo-hox]. % tämä ehkä muualle?





%% ----------------------------------
\section{Technical Details}
% HOX tää koko setti on lähinnä frontendissä, backendiin ei juurikaan tehty mitään lisätemppuja offline-moodia varten??
The technical details of the implementation covers almost entirely only the Kindergarten UI of the Päikky system. This is due to the allowed boundaries for the development team and the real life limitations of the Päikky system usage. In order to achieve the required level of offline support on the system, almost all of the work could have been done only on the mobile frontend codebase: the Kindergarten UI. In addition to the changes made to the backend, no other modules (\textit{Family UI, Manager UI}) of the system needed changes. However the other modules did not gain any kind of added offline support either.


%% ###
\subsection{HTTP Cache Headers}
% maininta, että tämä olisi ollut hyvä tehdä myös ilman offline-tuen tekemistä

% mahdollisia viitteitä:
% - http://tools.ietf.org/html/rfc2965 (????)
% - http://www.w3.org/Protocols/rfc2616/rfc2616-sec14.html#sec14.9 !

Starting point for the offline support implementation was to take as much advantage as possible from the techniques already in use on the Päikky system's implementation. First task for the development team was to ensure that the cache related features of the HTTP protocol were utilized thoroughly.

Previously there were issues reported by the users after production environment updates that the new features announced weren't visible on their devices. This was the result of poor and some part nonexistent cache header usage. The cache headers prior the change instructed the browser to save the \textit{index.html} document – the starting point of the Kindergarten UI web application – and all the JavaScript files for 24 hours on the file system of the device. If those files were requested within that 24 hours, the cached versions were used. This created a huge lag on the rollout of the latest version to the end users' devices, which caused work for the development team since both the old and the new version of the frontend client had to be supported on the backend.
% HOX tsekkaa tästä kappaleesta toi vanhan cachetusajan pituus, emmä muista, tuo 24h on vejetty stetsonista :----D 

The "release lag" was addressed by changing the HTTP 1.1 headers related to caching, returned by the Päikky backend's Apache web server. The first solution was to 
% päikky main.js -nykyrequsta:
\begin{lstlisting}
    Cache-Control:max-age=0, no-cache, no-store, must-revalidate
    Pragma:no-cache
    Date:Mon, 24 Nov 2014 08:36:04 GMT
    Expires:Wed, 11 Jan 1984 05:00:00 GMT
\end{lstlisting}
\lstinputlisting[caption=rororlrotl]

% päikky kuva-nykyrequsta:
% (86400s == 24h)
\begin{lstlisting}
    Cache-Control:max-age=86400
    Content-Type:image/jpeg
    Date:Mon, 24 Nov 2014 12:41:01 GMT
    Expires:Tue, 25 Nov 2014 12:41:01 GMT
\end{lstlisting}

Addressing the HTTP cache header related problems were not directly coupled to the implementation of offline mode, but fixing them would have benefit the Päikky system even if the offline support would not have been on the product's road map.


%% ###
\subsection{Application Cache}
% myös maininta, että tämä olisi ollut hyvä tehdä myös ilman offline-tuen tekemistä
Cache manifestista juttua, perusteet, ja se, miten Päikky sitä käyttää


%% ###
\subsection{Identifying Connection Issues}
Millä mekanismilla yhteyden häviäminen tunnistetaan?


%% ###
\subsection{Using Local Storage as a Cache}
Local storagen perusteet

REST-API-pyyntöjen cachettamisesta local storageen juttua



%% ###
\subsection{Job Queue: Promise-based Presence Marking queue}
Kuinka koodin tasolla on toteutettu tehtävien merkkauksien jonosysteemi ja sen purkaminen

(Perus)Teoriaa Promiseista JavaScriptissä


%% ###
\subsection{Storing and Receiving Presence Markings on the Server}
% ainoa offlineen liittyvä muutos mikä toteutettiin bäkkäriin
% presenceille timestamp, milloin ne on tehty -> voi (ja on) eri kuin mitä vastaanottotimestamp on
% -> yliyksinkertaistaen käytännössä kaikista merkinnöistä tuli "offlinessä-tehtyjä" verrattuna aikaisempaan, missä merkintöihin otettiin timestampit serverille saapumisajan mukaan




%% --------------
% \section{Known Limitations of the Offline mode}
% Esimerkiksi konfliktitilanteita nykyisessä ratkaisussa ei huomioida/niihin ei varauduta mitenkään (esimerkiksi jos kaksi käyttäjää yhtäaikaa editoi offlinessa juttuja ja laitteet menevät onlineen -> mitä tapahtuu on mysteeri. Tätä on paikattu käyttäjäkoulutuksella, että eivät aja järjestelmää tällaisiin tilanteisiin)


% (Onkohan tämä sittenkin turha kappale, tai väärässä paikassa? Olisiko yhdistettävissä evaluation-kappaleeseen?)
% --> toistaiseksi otettu pois, tätä kamaa tuli jo tämän chapterin alussa & lisää tulee (luultavasti) evaluationissa

