
\chapter{Materials and Methods}
Tämä kappale on itselleni kaikista hämärin, että mitä täällä pitäisi olla


%% ------------------------------------------
\section{Design Research}
Erityisesti tämä osio



%% ------------------------------------------
\section{Case: Päikky and Offline mode}
Tänne jotain tavaraa Päikky-toimitusjohtajan haastattelusta (jota ei siis vielä ole tehty?). Ehkä jos jotain löytyy kommunikaatiosta/kehitysdokumenteista niin se myös?

Kannattaisiko tänne kaivaa versionhallinnasta dataa offlinemoodista, tyyliin millä aikavälillä se kehitettiin, paljonko arvioitu rivien määrä, etc?

\subsection{Päikky System Description}
Application under our case analysis is Päikky, a solution for daycare personnel & daycare children’s parents for planning the needed daycare and passage control of the children on the daycare. 

Päikky consists of four parts:

\begin{enumerate}
	\item \textit{Kindergarten UI}, single-page application used for logging children in/out to kindergarten
	\item \textit{Family UI}, single-page application used by parents when planning children daycare time
	\item \textit{Manager UI}, single-page application used by kindergarten management 
	\item \textit{Backend}, Grails-powered REST-API
\end{enumerate}

\noindent In this thesis we are focusing on the \textit{Kindergarten UI}, which is the tool used on daily basis by the kindergarten nurses for tracking the attendance. From the Kindergarten UI nurses can also see elaborate information about the children, including allergies and persons who are allowed to pick them up, have parents allowed the children to be photographed. Updating this information is also possible. They can also monitor the overall presence of persons in the kindergarten, including both children and other nurses.

Key activity for nurses is to log in child when they arrive at the kindergarten and log them out when child leaves. To do this conveniently nurses are equipped with Android smart phones. This has to be done in order to replicate the presence status from the real world to the Päikky system. Based on the plans done by parents on the Family UI, nurses can see how many and who of the children they are expecting to appear for each day. This activity is repeated tens of times per day. Since monitoring attendance and creating bills for provided daycare is based on the presence data generated by logging the children in and out, it is crucial that this activity can be achieved successfully under any kind of condition.

%% jotain käyttöympäristöstä?

%% jotain käyttäjistä?



\subsection{Before and after Offline-mode}
Softan käytöstä ennen offline-modea

Softan käytöstä offline-moden jälkeen


% Implementaatiosta sen verran että tutkimusmetodina on implementoida teknologiaa ja tutkia kuinka se soveltuu arkipäivän käyttöön. Voisi myös miettiä lähestymistä että sulla on ollut kaksi eri tilannetta 1. softa ilman offline"palikkaa/featurea" ja 2. offline palikan kanssa. Tutkit että kuinka offlinepalikka vaikutti sovelluksen käyttökokemukseen.





%% ------------------------------------------
\section{Data Collection with Semi-Structured Interviews}
% MikkoKoski-dipasta saa hyvää mallia tähän

Validating Offline mode User Experience with User Interviews

Avattu, keneltä on kysytty (päiväkodin tädit) mitä (Päikky-järjestelmän käyttöä ja offline-moodin käsittämisestä juttua) ja miksi (pyritään ymmärtämään onko offline-mode vastaus todelliseen käyttäjien ongelmaan). Datan keräysmetodina puolistrukturoitu käyttäjähaastattelu.

\subsection{Finding Interviewees}
Ketä haastateltiin, millä perusteilla valittiin


\subsection{Preparing Interviews}
Miten haastattelurunko tehtiin


\subsection{Conducting Interviews}
Miten haastattelu toteutettiin


\subsection{Analyzing Interviews}
Miten analysointi tehtiin

% datan keräysmetodina käyttäjä/asiakashaastattelut (muuta?, kehitysdokumentit?)

  


