\chapter{Research Methods}
% Implementaatiosta sen verran että tutkimusmetodina on implementoida teknologiaa ja tutkia kuinka se soveltuu arkipäivän käyttöön. Voisi myös miettiä lähestymistä että sulla on ollut kaksi eri tilannetta 1. softa ilman offline"palikkaa/featurea" ja 2. offline palikan kanssa. Tutkit että kuinka offlinepalikka vaikutti sovelluksen käyttökokemukseen.

%% ------------------------------------------
\section{Design Research}
Tähän jotain... tiedettä... jostain...

"Konstruktiivinen tutkimus"
% Pyry-viitehox: Pertti Järvinen ja Annikki Järvinen - Tutkimustyön Metodeista

Piirainen-Gonzalez kirja Dropboxista


%% ------------------------------------------
\section{Data Collection with Semi-Structured Interviews}
% MikkoKoski-dipasta saa hyvää mallia tähän

"Validating Offline mode User Experience with User Interviews"

Avattu, keneltä on kysytty (päiväkodin tädit) mitä (Päikky-järjestelmän käyttöä ja offline-moodin käsittämisestä juttua) ja miksi (pyritään ymmärtämään onko offline-mode vastaus todelliseen käyttäjien ongelmaan). Datan keräysmetodina puolistrukturoitu käyttäjähaastattelu.

\subsection{Finding Interviewees}
Ketä haastateltiin, millä perusteilla valittiin


\subsection{Preparing Interviews}
Miten haastattelurunko tehtiin


\subsection{Conducting Interviews}
Miten haastattelu toteutettiin


\subsection{Analyzing Interviews}
Miten analysointi tehtiin

% datan keräysmetodina käyttäjä/asiakashaastattelut (muuta?, kehitysdokumentit?)

  