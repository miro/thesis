
\chapter{Evaluation}

\section{Technical Effectiviness of the Implementation}
Kuinka toimii teknisesti, toimiiko? Mitä huonoja puolia jäi?



%% -------------------------------
\section{User Experience on the Implementation}
Tätä mitattiin käyttäjähaastatteluilla, blabla, katso kappale Methods
% Validointi haastattelujen kautta, mitä mieltä offlinen toimivuudesta

\subsection{User interview results}
Selitystä haastiksien tuloksista. Saisko jopa jotain hassunhauskaa graafia aikaiseksi?



\subsection{Users' Understanding on Offline-mode}
- missä tieto säilössä? asiakaspalvelin-mallin ymmärtämättömyys




%% -------------------------------
\section{Design Guidelines}
Mitä suosituksia voisi Päikky-sekoilusta muodostaa jälkipolville?

\subsection{Recognize the Required Feature set for Offline Support}
Kaikkea ei välttämättä tarvitse tukea Offline-moodissa, resurssien säästämiseksi on mahdollista etsiä avainaktiviteetit ja mahdollistaa vain niiden käyttö


\subsection{Implement Offline Features in an Early πhase}
Jos on ajatus siitä, että voisi olla pienikin tarve offline-tuelle, tee se samantien (budjetin/ajan sallimissa rajoissa)

\subsection{Naming the Offline Concept on the UI-level}
Offline-moodi ei selkeä "normikäyttäjälle", nimeämistä tulee harkita
