\chapter{Evaluation}
% TODO:
% - offline-tilan tietojen cacheamis+tiedon enkryptaus -ongelma, siitä jotain? ei siis päikkyspesifi, mutta yleisesti. antin "no sehän on mahdoton ongelma"

\section{Technical Effectiviness of the Implementation}
% -> HTTP-cacheheaderit kuntoon, tehtiin offline-tukea tai ei?

% HOX tietoturvaongelmat tämän suhteen (ei tässä keississä mutta yleisesti), problematiikka tämän tiedon kryptaamisesta -> ei mahdollista (tätä hoxia on pastettu joka paikkaan, johonkin pitää selittää tästä, poista muut maininnat)

Kuinka toimii teknisesti, toimiiko? Mitä huonoja puolia jäi?



%% -------------------------------
\section{User Experience on the Implementation}
Tätä mitattiin käyttäjähaastatteluilla, blabla, katso kappale Methods
% Validointi haastattelujen kautta, mitä mieltä offlinen toimivuudesta

\subsection{User interview results}
Selitystä haastiksien tuloksista. Saisko jopa jotain hassunhauskaa graafia aikaiseksi?
%% pollausajan vaikutus lapsimäärien luotettavuuteen!


\subsection{Users' Understanding on Offline-mode}
- missä tieto säilössä? asiakaspalvelin-mallin ymmärtämättömyys




%% -------------------------------
\section{Design Guidelines}
% - AppCachen käyttösuosituksia
% - mahdolliset tietoturva&kryptausongelmat local storage -kaman kanssa?
Mitä suosituksia voisi Päikky-sekoilusta muodostaa jälkipolville?
Onko tämä osuus oikeassa chapterissa??

\subsection{Recognize the Required Feature set for Offline Support}
Kaikkea ei välttämättä tarvitse tukea Offline-moodissa, resurssien säästämiseksi on mahdollista etsiä avainaktiviteetit ja mahdollistaa vain niiden käyttö


\subsection{Implement Offline Features in an Early πhase}
Jos on ajatus siitä, että voisi olla pienikin tarve offline-tuelle, tee se samantien (budjetin/ajan sallimissa rajoissa)

\subsection{Naming the Offline Concept on the UI-level}
Offline-moodi ei selkeä "normikäyttäjälle", nimeämistä tulee harkita







%% -------------- TÄMÄ ON SIIRRETTY TOISTAISEKSI TÄNNE IMPLEMENTATION-CHAPTERISTA
% \section{Known Limitations of the Offline mode}
% Esimerkiksi konfliktitilanteita nykyisessä ratkaisussa ei huomioida/niihin ei varauduta mitenkään (esimerkiksi jos kaksi käyttäjää yhtäaikaa editoi offlinessa juttuja ja laitteet menevät onlineen -> mitä tapahtuu on mysteeri. Tätä on paikattu käyttäjäkoulutuksella, että eivät aja järjestelmää tällaisiin tilanteisiin)


% (Onkohan tämä sittenkin turha kappale, tai väärässä paikassa? Olisiko yhdistettävissä evaluation-kappaleeseen?)
% --> toistaiseksi otettu pois, tätä kamaa tuli jo tämän chapterin alussa & lisää tulee (luultavasti) evaluationissa

