\chapter{Evaluation}
% TODO:


%% -------------------------------
\section{Technical Effectiviness of the Implementation}
% -> HTTP-cacheheaderit kuntoon, tehtiin offline-tukea tai ei?

% HOX tietoturvaongelmat tämän suhteen (ei tässä keississä mutta yleisesti), problematiikka tämän tiedon kryptaamisesta -> ei mahdollista (tätä hoxia on pastettu joka paikkaan, johonkin pitää selittää tästä, poista muut maininnat)

Kuinka toimii teknisesti, toimiiko? Mitä huonoja puolia jäi?

Tarvittaiskohan tänne jotain mittausdataa? Mitä mittausdataa?

offline-tilan tietojen cacheamis+tiedon enkryptaus -ongelma, siitä jotain? ei siis päikkyspesifi, mutta yleisesti. antin "no sehän on mahdoton ongelma" (kryptausavaimen sijainti/toimitusongelma)
 



%% -------------------------------
\section{User Experience on the Offline support Implementation}
Tätä mitattiin käyttäjähaastatteluilla, blabla, katso kappale Methods
% Validointi haastattelujen kautta, mitä mieltä offlinen toimivuudesta

% ###
\subsection{User Interview Results}
Selitystä haastiksien tuloksista. Saisko jopa jotain hassunhauskaa graafia aikaiseksi?

-> pollausajan vaikutus lapsimäärien luotettavuuteen!
-> lasten merkkaus nopeampaa tilakonesiirtymän ansiosta, ei tarvitse odotella tuloksia serveriltä->pääsee nopeammin seuraavaan tehtävään

% ###
\subsection{Users' Understanding of the Offline mode}
- missä tieto säilössä? asiakaspalvelin-mallin ymmärtämättömyys



%% ###
\subsection{User Experience on the Limited Feature Set}
Esimerkiksi konfliktitilanteita nykyisessä ratkaisussa ei huomioida/niihin ei varauduta mitenkään (esimerkiksi jos kaksi käyttäjää yhtäaikaa editoi offlinessa juttuja ja laitteet menevät onlineen -> mitä tapahtuu on mysteeri. Tätä on paikattu käyttäjäkoulutuksella, että eivät aja järjestelmää tällaisiin tilanteisiin) 

