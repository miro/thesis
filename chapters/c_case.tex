
\chapter{Case Päikky}
todo blablabla


%% ------------------------------------------
\section{System Description}
TODO-LISTA:
-tänne Presence -termin avaus kunnolla! Siihen viitataan myöhemmin tyyliin jokapaikassa
-sekä lasten ETTÄ hoitajien läsnäoloa seurataan päikyllä!
-tänne Päikyn Presence-tilakone?
-tänne selitys Apachen ja Tomcatin suhteesta Päikyssä?
-Päikyn browser-scope avattuna tänne?
-REST API requista tänne jotain, että mitä hä? Hox linkissä tämä-implementation-research
-backbonesta jotain juttua? Vai backgroundiin?

This thesis approaches the research problem through a real-life case study on implementing an offline-mode to a mobile daycare application Päikky. Päikky is a collaborative mobile application for daycare personnel and children's parents to plan and coordinate daily activitities in kindergarten. 

%Application under our case analysis is Päikky, a solution for daycare personnel & daycare children’s parents for planning the needed daycare and passage control of the children on the daycare. 

Päikky consists of four parts:

\begin{enumerate}
	\item \textit{Kindergarten UI}, mobile-first single-page application used for logging children in/out to kindergarten
	\item \textit{Family UI}, single-page application used by parents when planning children daycare time
	\item \textit{Manager UI}, single-page application used by kindergarten management 
	\item \textit{Backend}, Grails-powered REST-API and back-end service
\end{enumerate}

\noindent In this thesis we are focusing on the \textit{Kindergarten UI}, which is the tool used on daily basis by the kindergarten nurses for tracking the attendance. From the Kindergarten UI nurses can also see elaborate information about the children, including allergies and persons who are allowed to pick them up, have parents allowed the children to be photographed. Updating this information is also possible. They can also monitor the overall presence of persons in the kindergarten, including both children and other nurses.

Key activity for nurses is to log in child when they arrive at the kindergarten and log them out when child leaves. To do this conveniently nurses are equipped with Android smart phones. This has to be done in order to replicate the presence status from the real world to the Päikky system. Based on the plans done by parents on the Family UI, nurses can see how many and who of the children they are expecting to appear for each day. This activity is repeated tens of times per day. Since monitoring attendance and creating bills for provided daycare is based on the presence data generated by logging the children in and out, it is crucial that this activity can be achieved successfully under any kind of condition.

-> jotain käyttöympäristöstä?

-> jotain käyttäjistä?

%% -------------------------------
\section{Offline Support}
tänne() taustoja budurajotteista, käytettävissä olevista henkilömääristä jne -> miksi tehtiin niinkin paljon kompromisseja kuin tehtiin

Tänne jotain tavaraa (mahdollisesta) Päikky-toimitusjohtajan haastattelusta (jota ei siis vielä ole tehty?). Ehkä jos jotain löytyy kommunikaatiosta/kehitysdokumenteista niin se myös?

Kannattaisiko tänne kaivaa versionhallinnasta dataa offlinemoodista, tyyliin millä aikavälillä se kehitettiin, paljonko arvioitu rivien määrä, etc? Vai voisko tämä knoppitieto olla jossain muualla?

\subsection{Before and after Offline-mode}

-> Softan käytöstä ennen offline-modea

-> Softan käytöstä offline-moden jälkeen







