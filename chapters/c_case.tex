
\chapter{Case Päikky}
This thesis approaches the research problem through a real-life case study where an offline support was implemented to a already live system. The system under the analysis is Päikky, a solution for daycare personnel & daycare children’s parents for planning the needed daycare and providing passage control of the children on the daycare. It also provides functionality which helps kindergarten personnel to coordinate daily activities on the kindergarten and communicate with the children's parents.


%% ------------------------------------------
\section{System Description}


Päikky is a collaborative mobile application for daycare personnel and children's parents to plan and coordinate daily activitities in kindergarten. 
%Application under our case analysis is Päikky, a solution for daycare personnel & daycare children’s parents for planning the needed daycare and passage control of the children on the daycare. 

Päikky consists of four parts:

\begin{enumerate}
	\item \textit{Kindergarten UI}, mobile-first single-page application used for logging children in/out to kindergarten, referenced as an \textit{application} later on
	\item \textit{Family UI}, single-page application used by parents when planning children daycare time
	\item \textit{Manager UI}, single-page application used by kindergarten management 
	\item \textit{backend}, Grails-powered server that implements REST API and other services used by the Päikky UI's 
\end{enumerate}

\noindent In this thesis we are focusing on the \textit{Kindergarten UI}, which is the tool used on daily basis by the kindergarten nurses for tracking the attendance on their care group. 

<TODO: kaaviokuva yleissetupista? Vai onko vain redundanssia?>

% ##
\subsection{Backend}

The backend is the web server of Päikky which holds all the master data of the system. It serves the HTML files and other resources (graphical assets, JavaScript files, Cascading Style Sheets) to the other modules of Päikky. The backend also implements a REST API which from the other modules requests and alters the data on the system. Data is stored in a \textit{MySQL} database.

Currently the backend is operated as a single virtual machine on the Amazon Web Services cloud. The backend is \textif{multitenant}: single instance serves multiple municipalities and kindergartens. % hox todo tänne sellanen numeroviittaus

%-tänne selitys Apachen ja Tomcatin suhteesta Päikyssä?

%-REST API requista tänne jotain, että mitä hä? Hox linkissä tämä-implementation-research (vai tulisko tää tohon ylempään sectioniin?)

<TODO tarvisko jotain lisää kertoa vielä bäkkäristä? mitä?>

% ##
\subsection{Kindergarten UI}

Kindergarten UI is the part of Päikky which is used daily by the kindergarten nurses. From the Kindergarten UI nurses can also see elaborate information about the children, including allergies and persons who are allowed to pick them up, have parents allowed the children to be photographed. Nurses can also easily see the contact information for each child. Updating this information is also possible. %They can also monitor the overall presence of persons in the kindergarten, including both children and other nurses.

On Kindergarten UI the key activity for nurses is to log in children (and the other nurses) when they arrive at the kindergarten and log them out when a child or nurse leaves. This activity is repeated tens of times per day. To do this conveniently nurses are equipped with \textit{Android} smart phones. The marking of goings and leavings has to be done in order to replicate the presence status from the real world to the Päikky system. Since monitoring attendance and in the future creating bills for provided daycare is based on the presence data generated by logging the personnel in and out, it is crucial that this activity can be achieved successfully under any kind of condition. Doing this activity should also be as effortless and simple as possible for the nurses so that it would get done at the exact moment when the actual event happens on the kindergarten. 

The simplicity requirement goes for every other aspect of the system: the Päikky users' demography is a very mixed crowd. The kindergarten nurses' age can be anything from 18 to 65. Because of the age variation also the ability and the starting level to use a digital service via smartphone varies a lot. Taking the easiness of usage is also one of the key principles behind Päikky's user interface and interaction design. It is also one of the unique selling points of the company behind Päikky, the \textit{MukavaIT}: "using IT systems shouldn't be hard or unpleasant" [viite mukavait.fi].

Based on the plans done by parents on the \textit{Family UI}, nurses can see how many and who of the children they are expecting to appear for each day. If the parents have to change the already existing plans with short notice, automated message is sent to the nurses stating the change, and the plans visible for the child relating the case gets updated in real time. Nurses can also see the exact amount of children and nurses present at the kindergarten for any given time.

The usual usage environment for Kindergarten UI are the kindergarten

From technical point of view looking the Kindergarten UI is a single-page web application (explained on the <chapter two> [todo hox viittaus wadap]) created to be used primarily with mobile devices. The devices and browsers targeted during the development process were \textit{Samsung Ace II}'s and the latest stable version of the \textit{Chrome}. The libraries the application consists of are: % todo hox tarvitaanko viitettä tohon samsungii? todo tänne semmonen alaviite myös vai?

\begin{enumerate}
	\item \textit{Backbone}, \textit{Model-View-Template} –framework providing the – wait for it – backbone for the application,
	\item \textit{Marionette}, library of common design and implementation patterns for Backbone,
	\item \textit{Require}, module loader and dependency manager,
	\item \textit{Underscore}, functional programming inspired library for utility functions,
	\item \textit{Moment}, time and timezone handling library. 
\end{enumerate}

<TODO: tänne semmonen numeroviittaus noihin kirjastoihin ja sivun alalaitaan linkit, vai miten?>





% ##
\subsection{Presence Model and the Presence State Machine}
% -tänne Päikyn Presence-tilakone kaavion kera
% -tänne Presence -termin avaus kunnolla! Siihen viitataan myöhemmin tyyliin jokapaikassa
% -termien "Presence type" ja "log marking" (tms) avaaminen
% -> Presence type/state (kumpi?!), Presence state transfer tms, check implementaation viimeinen kappale


<KUVA-TODO: Presence State Machine>




%% -------------------------------
\section{Emerged Need for Offline Support}
tänne() taustoja budurajotteista, käytettävissä olevista henkilömääristä jne 
-> miksi tehtiin niinkin paljon kompromisseja kuin tehtiin (budu, MVP-lähestyminen)
-> Pete-haastiksesta (mukavait tj) kamaa

% Tänne jotain tavaraa (mahdollisesta) Päikky-toimitusjohtajan haastattelusta (jota ei siis vielä ole tehty?). Ehkä jos jotain löytyy kommunikaatiosta/kehitysdokumenteista niin se myös?

Kannattaisiko tänne kaivaa versionhallinnasta dataa offlinemoodista, tyyliin millä aikavälillä se kehitettiin, paljonko arvioitu rivien määrä, etc? Vai voisko tämä knoppitieto olla jossain muualla? Tai ylipäätään missään?


-> Softan käytöstä ennen/jälkeen offline-modea (onkohan tää ny turhaa asiaa ylipäätään?)




% ##
% \subsection{Before and after Offline-mode}
% tämä kappale on toistaiseksi hyllytetty






