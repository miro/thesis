\chapter{Lessons Learned}

This chapter introduces lessons learnt by the developers while implementing the offline support to Päikky. 

The points presented on the \textit{design guidelines} section are factors that are recommended to be taken into account when developing similar functionalities into other single-page applications.

The discussion section speculates with points and matters detected during the development and deployment of the offline support, but which were too vague to be based any recommendations on. 




%% -------------------------------
\section{Design Guidelines}
%"Mitä suosituksia voisi Päikky-sekoilusta muodostaa jälkipolville?""

%- AppCachen käyttösuosituksia
%- mahdolliset tietoturva&kryptausongelmat local storage -kaman kanssa?



% ###
\subsection{Recognize the Essential Feature Set for Offline Support}
% Kaikkea ei välttämättä tarvitse tukea Offline-moodissa, resurssien säästämiseksi on mahdollista etsiä avainaktiviteetit ja mahdollistaa vain niiden käyttö

% ### -> käytä appcachea yms vaikka ei olekaan tarkoituksena tehdä varsinaista offline-tukea(????)


% ###
\subsection{Prepare for Possible Offline Support}
% Jos on ajatus siitä, että voisi olla pienikin tarve offline-tuelle, tee se samantien (budjetin/ajan sallimissa rajoissa)

% päikyssä myöhäisestä implmeentoinnista ei varsinaisesti ollut haittaa, mutta olisi ehkä nopeuttanut  kehitystä

% -> datasynkkaus selaimessa yhteen keskitettyyn paikkaan, mistä se on helppo kirjoittaa suuntaan tai toiseen (päikyssä backbone.sync)





% ###
\subsection{Naming the Offline Concept on the UI-level}
% Offline-moodi ei selkeä "normikäyttäjälle", nimeämistä tulee harkita, viite discussioniin
% discussioniin?



%% -------------------------------
\section{Discussion}

<spekulaatiota siitä, pitäisikö opettaa käyttäjille asiakas-palvelin-systeemin periaatteet>


-> väite: offline-hommat tulevat käytännössä pakollisiksi lähituleviasuudessa vakavastiotettavissa web-applikaatioissa





% ---Mitä opittiin/tehtiin---

% Further research on the topic could focus more on the human aspects of the development and ignore the traditional software quality assurance. 
