\chapter{Lessons Learned}
todo bla bla bla



%% -------------------------------
\section{Design Guidelines}
- AppCachen käyttösuosituksia
- mahdolliset tietoturva&kryptausongelmat local storage -kaman kanssa?
"Mitä suosituksia voisi Päikky-sekoilusta muodostaa jälkipolville?""

\subsection{Recognize the Feature Set Required for Offline Support}
Kaikkea ei välttämättä tarvitse tukea Offline-moodissa, resurssien säästämiseksi on mahdollista etsiä avainaktiviteetit ja mahdollistaa vain niiden käyttö


\subsection{Implement Offline Features in an Early phase}
Jos on ajatus siitä, että voisi olla pienikin tarve offline-tuelle, tee se samantien (budjetin/ajan sallimissa rajoissa)
% päikyssä myöhäisestä implmeentoinnista ei varsinaisesti ollut haittaa, mutta olisi ehkä nopeuttanut  kehitystä

\subsection{Naming the Offline Concept on the UI-level}
Offline-moodi ei selkeä "normikäyttäjälle", nimeämistä tulee harkita



%% -------------------------------
\section{Discussion}

<spekulaatiota siitä, pitäisikö opettaa käyttäjille asiakas-palvelin-systeemin periaatteet>


-> väite: offline-hommat tulevat käytännössä pakollisiksi lähituleviasuudessa vakavastiotettavissa web-applikaatioissa




% HUOM nämä mpohj kommentteja

% kerrottiin softa-startupeista yleisellä tasolla
% listattiin perinteisen softakehityksen laatumetodit eri elinkaaren vaiheissa sekä metodien tehokkuudesta
% kerrottiin laadusta yleisesti sekä startupin näkökulmasta
% kerrottiin startupin laatukäsityksestä ja siitä miten hyvää laatua voi tavoitella
% yhdistettiin perinteisen softakehityksen laatumetodeita startupin ympäristöön
% käsiteltiin case-projektin suoritus ja vaiheet 
% käsiteltiin case-projektin toteutunut laatu haastattelujen perusteella



% ---Mitä opittiin/tehtiin---

% Further research on the topic could focus more on the human aspects of the development and ignore the traditional software quality assurance. 
