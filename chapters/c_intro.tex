\chapter{Introduction}
%% TODO:
% - pähkinänkuoressa koko setti tänne!
% - tästä hyötyä? http://dl.acm.org/citation.cfm?id=2307649
% - pitäiskö tänne lisätä että Kindergarten UI:n viitataan yleisesti applikaationa tässä viritelmässä?



% ### yleistä taustaa (esim. on enemmän ja enemmän verkon kautta käytettäviä mobiilipalveluita. tämä vaatii sen että verkko toimii, ihmisten arki ja sen vuorovaikutustilanteen ovat nykyään hyvin verkkovälitteisiä)
% Maailma digitalisoituu ympärillämme, yhä useampi työtehtävistä ja vapaa-ajan aktiviteeteista liittyy jotenkin digitaalisiin palveluihin ja laitteisiin. Harva näistä toimii täysin ilman Internetin tukea
The world is digitalizing around us on a always quicker pace. More and more of our work tasks and everyday activities are affiliated to digital devices and services. Most of these things are often useless without connectivity to a broader context, usually the Internet.

% Mobile usage has surpassed desktop usage in <lähde>, on jo iso joukko käyttäjiä joiden ainoa tie Internetiin on heidän mobiililaitteensa <lähde>. Joku viittaus "seuraavaan miljardiin netin käyttäjään?""

Age of the desktop hegemony is already over. Globally over a third of the Internet usage is done with mobile devices (smartphones and tablets) \cite{statcounter_global_????}. Unlike as in desktop devices, which usually are operated from static location aided by physical access to the Internet, mobile devices often travel with their owners where ever they might go. That creates challenges for connectivity, which is nowadays essential for device proper usage.

% Mobiilinetti-coverage ei optimaalinen -> verkkokatkoksia tulee, laadullisia ja kokonaan yhteyden katoamisia. Näiltä ei voida välttyä.
The problem with mobile devices is that mobile broadband coverage varies within different locations. When mobile device is not in the coverage area of a Wireless Local Area Network, device relies on a possible mobile broadband connection. When moving away from the nearest base station the reception of signal weakens. Mobile device user experiences this as an increased latency times, lower data transferring speeds, or even as a total blackout of the broadband service. When moving away from highly populated areas this is inevitable, since not all locations are to be covered properly while keeping cost efficiency on a sane level, especially in countries with large acreage but low population, like Finland. However digitalization of services is even more important on a places like this, since supplying physical services to areas with low population density is expensive. 

%\section{Motivation}
This makes digital services more and more common in business critical tasks for a emerging amount of applications. But how to use services and devices dependent on the Internet with a coincidental connection? With the increasing mobile device usage, adding offline support at least in some level will not be an uncommon task for any web developer.

The Internet is build on top of the client-server -paradigm \cite{berson_client-server_1992}. This means that the server holds the master data, and clients has to do all the data fetching and modifying through the server. Without connectivity to the Internet, the client has to manage with the content it might already have loaded. This is the basic problem surrounding connectivity issues -- without access to the Internet, there is not much what the client can do.






\section{Context}
%The Internet is build on top of the client-server --paradigm \cite{berson_client-server_1992}. In this thesis we focus mainly on the client end of this paradigm and research ways of dealing the connectivity problem in there while delimiting the server out of this thesis' context. 

This thesis studies one real-life case of implementing offline support to already existing system created for supporting kindergarten personnel and the parents of daycare children, the Päikky system. Päikky is used by various kindergartens in Finland and its key feature is to provide passage control mechanism for the kindergarten's children.

The Päikky system is owned and managed by \textit{MukavaIT Oy} (later referred to as the \textit{client organization} on this thesis), a Finnish start-up company. The client organization has bought consulting and development to  Päikky from \textit{Futurice Oy}, where the author of this thesis has also worked as a part of the Päikky development team.

On a high level Päikky consists of four modules: \textit{Family UI, Kindergarten UI, Manager UI} and the \textit{backend}. Under this thesis' analysis is especially the \textit{Kindergarten UI}, which is a mobile-first designed web application. Kindergarten nurses uses the Kindergarten UI on smartphones every time a child enters or leaves the kindergarten. As this is an additional task to their current workload, it has to be as effortless as possible, especially when it is performed tens of times per day. % IMPR tänne selitystä/viite mobile firstiin?

The Kindergarten UI is a web application built in accorddance with the \textit{single-page application} paradigm. In contrary to traditional web pages, where the server outputs the asked content as a structured and stylized HTML form, single-page app uses mainly \textit{REST APIs} for the exchange of the data. With REST API only the information can transfer between the client and the browser, making the client in charge for the rendering format and style of the data. This results in a business logic being duplicated on the browser, making the client more ``fatter''. And since the logic exists side by side on the client as well as on the server, the client can be prepared to operate even when the connectivity to the server is temporarily lost.

In this thesis the emphasis will be on how to deal with the connection issue on inside the browser environment, ignoring the possible solutions that could be done on the device platform tier. Therefore results of this thesis will apply browsers in desktop computers, tablet devices and mobile phones, while leaving the network level solutions out of the scope.



%\section{Scope}



\section{Research Objectives}
This thesis researches an implementation of an offline mode to an existing single-page application, and studies the possible missteps that were made. Ultimately this thesis aims to define design guidelines for both in user experience and in software development -wise, so other developers would be able to achieve offline support implementation with less work and potentially with better results.

The objective of this research is to find useful and working fallback mechanisms for connectivity loss in single-page web applications, in both from the user experience perspective and from the developer's viewpoint. The state where the application loses the connectivity to the Internet is referred on this thesis as an \textbf{offline mode}. In this thesis an artifact is implemented and evaluated which allows the user to continue their tasks when the connection is lost and the application enters the offline mode.

From software development point of view the question is, how to deliver seamless user experience, even when the connectivity is bad or nonexistent? How much resources can this reserve on the development phase? Is it possible to abstract the connection quality completely away and handle the issues with connection internally on the application level, invisible to the user?

From the user experience perspective thesis concentrates on how to notify the user about the loss of Internet connection. Does the user need to be notified at all? If the user is notified, what is exactly the message that needs to be told? 

%\section{Research Question}
This thesis answers to the following question: 
\begin{quote}
	\textit``How to develop fallback mechanisms efficiently for Internet connection shortages on a single-page web application to support user experience''}
\end{quote}

Based on the research, implementation and evaluation done on this thesis, a recommendations for other developers are presented on the form of design guidelines. These guidelines are worth taking into consideration when creating any kind of single-page application which is aimed to be used with mobile devices.




\section{Structure of the thesis}
% In here well be described contents of each chapter, because that's how it has always been done.

\textit{Chapter 2} opens up the background in which from the themes of this thesis are originating from. The research gap in which this thesis resides is also introduced. \textit{Chapter 3} describes the real-life case which is in a vital role on this thesis. Parts that are essential for understanding the concepts discussed on this thesis are opened thoroughly. \textit{Chapter 4} introduces the methods used to conduct the research done on this thesis. On \textit{Chapter 5} technical solutions implemented are summed up and described. On \textit{Chapter 6} these solutions are evaluated, from both technical viewpoint and from the user experience perspective. Evaluation is done based on user interviews conducted during this thesis. \textit{Chapter 7} presents the design guidelines formed based on the evaluation done on the previous chapters. Also aspects noted during the evaluation but which were too indeterminate to be design guidelines are discussed on the same chapter. Conclusions done based on the implementation and the evaluation of the offline support are presented on the \textit{Chapter 8}.




%% Tommi-input: 
% "rajaus pitää selittää johdannossa,tiivistössä ja abstraktissa"
% -mihin liittyy, konteksti
% -ongelma, mitä tehdään kun yhteys katoaa
% -ratkaisu, “tässä diplomityössä tehdään”
% -arviointi, “miten hyvä tästä tuli”




% ### - ongelmakentän kuvaus, mikä on tutkimusta vaativa ongelma (kentän katkeaminen ja minkälaisia ongemia se aiheuttaa) ja se ajankohtaisuus ja yleisyys. Ongelma on että ei tiedetä kovin hyvin että kuinka kentän katkeamisen vaikutukset pitäisi ottaa sovelluskehityksessä huomioon niin että käyttäjän tavoitteet sovelluksen käytölle täyttyvät kentän katkeamisesta huolimatta. 






% Tässä dipassa ongelmaa lähdetään selvittämään case studyllä liittyen Päikky-järjestelmään. <Pieni pohjustus järjestelmästä>
% - kerro miten lähdet tässä tutkimuksessa ratkaisemaan ongelmaa. Case tutkimus päiväkotisoftalle. Tehtiin päiväkotisofta, mutta huomattiin että verkonkatkeilu muodostaa merkittävän ongelman (kerro myös konkreettisesti mitä ongelmia katkeilu synnytti). Sitä ongelmaa lähdettiin sitten ratkaisemaan. Softan implementointi ja real life kenttäkoe implementoinnin vaikutuksista 


% <Päikyn ei-offline-versiossa ilmenneet ongelmat>

% <Lyhyesti offline-moden implementoinnista ja tosimaailman kenttäkokeesta>

% Offline-moden vaikutukset käytössä <haastatteluista rojua tähän>

% META:
% pitäisikö intron olla oma kokonaisuus? tarvitsisiko myöhemmissä kappaleissa vielä erikseen avata päikky-järjestelmää, vai voiko introssa käydä perinpohjaisesti läpi päikyn ja sitte that's it? Kävi myös mielessä pitäisikö ennen releated researchia olla yksi kappale nimenomaan Päikky-järjestelmän käytöstä
